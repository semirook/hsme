% Generated by Sphinx.
\def\sphinxdocclass{report}
\documentclass[letterpaper,10pt,english]{sphinxmanual}
\usepackage[utf8]{inputenc}
\DeclareUnicodeCharacter{00A0}{\nobreakspace}
\usepackage{cmap}
\usepackage[T1]{fontenc}
\usepackage{babel}
\usepackage{times}
\usepackage[Bjarne]{fncychap}
\usepackage{longtable}
\usepackage{sphinx}
\usepackage{multirow}


\title{hsme Documentation}
\date{March 17, 2014}
\release{0.1}
\author{Roman Semirook}
\newcommand{\sphinxlogo}{}
\renewcommand{\releasename}{Release}
\makeindex

\makeatletter
\def\PYG@reset{\let\PYG@it=\relax \let\PYG@bf=\relax%
    \let\PYG@ul=\relax \let\PYG@tc=\relax%
    \let\PYG@bc=\relax \let\PYG@ff=\relax}
\def\PYG@tok#1{\csname PYG@tok@#1\endcsname}
\def\PYG@toks#1+{\ifx\relax#1\empty\else%
    \PYG@tok{#1}\expandafter\PYG@toks\fi}
\def\PYG@do#1{\PYG@bc{\PYG@tc{\PYG@ul{%
    \PYG@it{\PYG@bf{\PYG@ff{#1}}}}}}}
\def\PYG#1#2{\PYG@reset\PYG@toks#1+\relax+\PYG@do{#2}}

\expandafter\def\csname PYG@tok@gd\endcsname{\def\PYG@tc##1{\textcolor[rgb]{0.63,0.00,0.00}{##1}}}
\expandafter\def\csname PYG@tok@gu\endcsname{\let\PYG@bf=\textbf\def\PYG@tc##1{\textcolor[rgb]{0.50,0.00,0.50}{##1}}}
\expandafter\def\csname PYG@tok@gt\endcsname{\def\PYG@tc##1{\textcolor[rgb]{0.00,0.27,0.87}{##1}}}
\expandafter\def\csname PYG@tok@gs\endcsname{\let\PYG@bf=\textbf}
\expandafter\def\csname PYG@tok@gr\endcsname{\def\PYG@tc##1{\textcolor[rgb]{1.00,0.00,0.00}{##1}}}
\expandafter\def\csname PYG@tok@cm\endcsname{\let\PYG@it=\textit\def\PYG@tc##1{\textcolor[rgb]{0.25,0.50,0.56}{##1}}}
\expandafter\def\csname PYG@tok@vg\endcsname{\def\PYG@tc##1{\textcolor[rgb]{0.73,0.38,0.84}{##1}}}
\expandafter\def\csname PYG@tok@m\endcsname{\def\PYG@tc##1{\textcolor[rgb]{0.13,0.50,0.31}{##1}}}
\expandafter\def\csname PYG@tok@mh\endcsname{\def\PYG@tc##1{\textcolor[rgb]{0.13,0.50,0.31}{##1}}}
\expandafter\def\csname PYG@tok@cs\endcsname{\def\PYG@tc##1{\textcolor[rgb]{0.25,0.50,0.56}{##1}}\def\PYG@bc##1{\setlength{\fboxsep}{0pt}\colorbox[rgb]{1.00,0.94,0.94}{\strut ##1}}}
\expandafter\def\csname PYG@tok@ge\endcsname{\let\PYG@it=\textit}
\expandafter\def\csname PYG@tok@vc\endcsname{\def\PYG@tc##1{\textcolor[rgb]{0.73,0.38,0.84}{##1}}}
\expandafter\def\csname PYG@tok@il\endcsname{\def\PYG@tc##1{\textcolor[rgb]{0.13,0.50,0.31}{##1}}}
\expandafter\def\csname PYG@tok@go\endcsname{\def\PYG@tc##1{\textcolor[rgb]{0.20,0.20,0.20}{##1}}}
\expandafter\def\csname PYG@tok@cp\endcsname{\def\PYG@tc##1{\textcolor[rgb]{0.00,0.44,0.13}{##1}}}
\expandafter\def\csname PYG@tok@gi\endcsname{\def\PYG@tc##1{\textcolor[rgb]{0.00,0.63,0.00}{##1}}}
\expandafter\def\csname PYG@tok@gh\endcsname{\let\PYG@bf=\textbf\def\PYG@tc##1{\textcolor[rgb]{0.00,0.00,0.50}{##1}}}
\expandafter\def\csname PYG@tok@ni\endcsname{\let\PYG@bf=\textbf\def\PYG@tc##1{\textcolor[rgb]{0.84,0.33,0.22}{##1}}}
\expandafter\def\csname PYG@tok@nl\endcsname{\let\PYG@bf=\textbf\def\PYG@tc##1{\textcolor[rgb]{0.00,0.13,0.44}{##1}}}
\expandafter\def\csname PYG@tok@nn\endcsname{\let\PYG@bf=\textbf\def\PYG@tc##1{\textcolor[rgb]{0.05,0.52,0.71}{##1}}}
\expandafter\def\csname PYG@tok@no\endcsname{\def\PYG@tc##1{\textcolor[rgb]{0.38,0.68,0.84}{##1}}}
\expandafter\def\csname PYG@tok@na\endcsname{\def\PYG@tc##1{\textcolor[rgb]{0.25,0.44,0.63}{##1}}}
\expandafter\def\csname PYG@tok@nb\endcsname{\def\PYG@tc##1{\textcolor[rgb]{0.00,0.44,0.13}{##1}}}
\expandafter\def\csname PYG@tok@nc\endcsname{\let\PYG@bf=\textbf\def\PYG@tc##1{\textcolor[rgb]{0.05,0.52,0.71}{##1}}}
\expandafter\def\csname PYG@tok@nd\endcsname{\let\PYG@bf=\textbf\def\PYG@tc##1{\textcolor[rgb]{0.33,0.33,0.33}{##1}}}
\expandafter\def\csname PYG@tok@ne\endcsname{\def\PYG@tc##1{\textcolor[rgb]{0.00,0.44,0.13}{##1}}}
\expandafter\def\csname PYG@tok@nf\endcsname{\def\PYG@tc##1{\textcolor[rgb]{0.02,0.16,0.49}{##1}}}
\expandafter\def\csname PYG@tok@si\endcsname{\let\PYG@it=\textit\def\PYG@tc##1{\textcolor[rgb]{0.44,0.63,0.82}{##1}}}
\expandafter\def\csname PYG@tok@s2\endcsname{\def\PYG@tc##1{\textcolor[rgb]{0.25,0.44,0.63}{##1}}}
\expandafter\def\csname PYG@tok@vi\endcsname{\def\PYG@tc##1{\textcolor[rgb]{0.73,0.38,0.84}{##1}}}
\expandafter\def\csname PYG@tok@nt\endcsname{\let\PYG@bf=\textbf\def\PYG@tc##1{\textcolor[rgb]{0.02,0.16,0.45}{##1}}}
\expandafter\def\csname PYG@tok@nv\endcsname{\def\PYG@tc##1{\textcolor[rgb]{0.73,0.38,0.84}{##1}}}
\expandafter\def\csname PYG@tok@s1\endcsname{\def\PYG@tc##1{\textcolor[rgb]{0.25,0.44,0.63}{##1}}}
\expandafter\def\csname PYG@tok@gp\endcsname{\let\PYG@bf=\textbf\def\PYG@tc##1{\textcolor[rgb]{0.78,0.36,0.04}{##1}}}
\expandafter\def\csname PYG@tok@sh\endcsname{\def\PYG@tc##1{\textcolor[rgb]{0.25,0.44,0.63}{##1}}}
\expandafter\def\csname PYG@tok@ow\endcsname{\let\PYG@bf=\textbf\def\PYG@tc##1{\textcolor[rgb]{0.00,0.44,0.13}{##1}}}
\expandafter\def\csname PYG@tok@sx\endcsname{\def\PYG@tc##1{\textcolor[rgb]{0.78,0.36,0.04}{##1}}}
\expandafter\def\csname PYG@tok@bp\endcsname{\def\PYG@tc##1{\textcolor[rgb]{0.00,0.44,0.13}{##1}}}
\expandafter\def\csname PYG@tok@c1\endcsname{\let\PYG@it=\textit\def\PYG@tc##1{\textcolor[rgb]{0.25,0.50,0.56}{##1}}}
\expandafter\def\csname PYG@tok@kc\endcsname{\let\PYG@bf=\textbf\def\PYG@tc##1{\textcolor[rgb]{0.00,0.44,0.13}{##1}}}
\expandafter\def\csname PYG@tok@c\endcsname{\let\PYG@it=\textit\def\PYG@tc##1{\textcolor[rgb]{0.25,0.50,0.56}{##1}}}
\expandafter\def\csname PYG@tok@mf\endcsname{\def\PYG@tc##1{\textcolor[rgb]{0.13,0.50,0.31}{##1}}}
\expandafter\def\csname PYG@tok@err\endcsname{\def\PYG@bc##1{\setlength{\fboxsep}{0pt}\fcolorbox[rgb]{1.00,0.00,0.00}{1,1,1}{\strut ##1}}}
\expandafter\def\csname PYG@tok@kd\endcsname{\let\PYG@bf=\textbf\def\PYG@tc##1{\textcolor[rgb]{0.00,0.44,0.13}{##1}}}
\expandafter\def\csname PYG@tok@ss\endcsname{\def\PYG@tc##1{\textcolor[rgb]{0.32,0.47,0.09}{##1}}}
\expandafter\def\csname PYG@tok@sr\endcsname{\def\PYG@tc##1{\textcolor[rgb]{0.14,0.33,0.53}{##1}}}
\expandafter\def\csname PYG@tok@mo\endcsname{\def\PYG@tc##1{\textcolor[rgb]{0.13,0.50,0.31}{##1}}}
\expandafter\def\csname PYG@tok@mi\endcsname{\def\PYG@tc##1{\textcolor[rgb]{0.13,0.50,0.31}{##1}}}
\expandafter\def\csname PYG@tok@kn\endcsname{\let\PYG@bf=\textbf\def\PYG@tc##1{\textcolor[rgb]{0.00,0.44,0.13}{##1}}}
\expandafter\def\csname PYG@tok@o\endcsname{\def\PYG@tc##1{\textcolor[rgb]{0.40,0.40,0.40}{##1}}}
\expandafter\def\csname PYG@tok@kr\endcsname{\let\PYG@bf=\textbf\def\PYG@tc##1{\textcolor[rgb]{0.00,0.44,0.13}{##1}}}
\expandafter\def\csname PYG@tok@s\endcsname{\def\PYG@tc##1{\textcolor[rgb]{0.25,0.44,0.63}{##1}}}
\expandafter\def\csname PYG@tok@kp\endcsname{\def\PYG@tc##1{\textcolor[rgb]{0.00,0.44,0.13}{##1}}}
\expandafter\def\csname PYG@tok@w\endcsname{\def\PYG@tc##1{\textcolor[rgb]{0.73,0.73,0.73}{##1}}}
\expandafter\def\csname PYG@tok@kt\endcsname{\def\PYG@tc##1{\textcolor[rgb]{0.56,0.13,0.00}{##1}}}
\expandafter\def\csname PYG@tok@sc\endcsname{\def\PYG@tc##1{\textcolor[rgb]{0.25,0.44,0.63}{##1}}}
\expandafter\def\csname PYG@tok@sb\endcsname{\def\PYG@tc##1{\textcolor[rgb]{0.25,0.44,0.63}{##1}}}
\expandafter\def\csname PYG@tok@k\endcsname{\let\PYG@bf=\textbf\def\PYG@tc##1{\textcolor[rgb]{0.00,0.44,0.13}{##1}}}
\expandafter\def\csname PYG@tok@se\endcsname{\let\PYG@bf=\textbf\def\PYG@tc##1{\textcolor[rgb]{0.25,0.44,0.63}{##1}}}
\expandafter\def\csname PYG@tok@sd\endcsname{\let\PYG@it=\textit\def\PYG@tc##1{\textcolor[rgb]{0.25,0.44,0.63}{##1}}}

\def\PYGZbs{\char`\\}
\def\PYGZus{\char`\_}
\def\PYGZob{\char`\{}
\def\PYGZcb{\char`\}}
\def\PYGZca{\char`\^}
\def\PYGZam{\char`\&}
\def\PYGZlt{\char`\<}
\def\PYGZgt{\char`\>}
\def\PYGZsh{\char`\#}
\def\PYGZpc{\char`\%}
\def\PYGZdl{\char`\$}
\def\PYGZhy{\char`\-}
\def\PYGZsq{\char`\'}
\def\PYGZdq{\char`\"}
\def\PYGZti{\char`\~}
% for compatibility with earlier versions
\def\PYGZat{@}
\def\PYGZlb{[}
\def\PYGZrb{]}
\makeatother

\begin{document}

\maketitle
\tableofcontents
\phantomsection\label{index::doc}


Содержание:


\chapter{Описание таблицы переходов в XML}
\label{xml_format:xml}\label{xml_format::doc}\label{xml_format:hsme-lotus}\label{xml_format:xml-format}
Реальный пример описания таблицы переходов с callback{}`ами для реализации логики компонента ``Корзина'':

\begin{Verbatim}[commandchars=\\\{\}]
\PYG{n+nt}{\PYGZlt{}scxml} \PYG{n+na}{version=}\PYG{l+s}{\PYGZdq{}1.0\PYGZdq{}}\PYG{n+nt}{\PYGZgt{}}

    \PYG{n+nt}{\PYGZlt{}state} \PYG{n+na}{id=}\PYG{l+s}{\PYGZdq{}in\PYGZus{}recalculation\PYGZdq{}} \PYG{n+na}{targetns=}\PYG{l+s}{\PYGZdq{}charts.basket\PYGZus{}callbacks.in\PYGZus{}recalculation\PYGZdq{}}\PYG{n+nt}{\PYGZgt{}}
        \PYG{n+nt}{\PYGZlt{}onentry} \PYG{n+na}{target=}\PYG{l+s}{\PYGZdq{}on\PYGZus{}enter\PYGZus{}in\PYGZus{}recalculation\PYGZdq{}}\PYG{n+nt}{/\PYGZgt{}}
        \PYG{n+nt}{\PYGZlt{}onchange} \PYG{n+na}{target=}\PYG{l+s}{\PYGZdq{}on\PYGZus{}change\PYGZus{}in\PYGZus{}recalculation\PYGZdq{}}\PYG{n+nt}{/\PYGZgt{}}
        \PYG{n+nt}{\PYGZlt{}onexit} \PYG{n+na}{target=}\PYG{l+s}{\PYGZdq{}on\PYGZus{}exit\PYGZus{}in\PYGZus{}recalculation\PYGZdq{}}\PYG{n+nt}{/\PYGZgt{}}

        \PYG{n+nt}{\PYGZlt{}state} \PYG{n+na}{id=}\PYG{l+s}{\PYGZdq{}in\PYGZus{}basket\PYGZus{}normal\PYGZdq{}} \PYG{n+na}{targetns=}\PYG{l+s}{\PYGZdq{}charts.basket\PYGZus{}callbacks.in\PYGZus{}basket\PYGZus{}normal\PYGZdq{}}\PYG{n+nt}{\PYGZgt{}}
            \PYG{n+nt}{\PYGZlt{}onentry} \PYG{n+na}{target=}\PYG{l+s}{\PYGZdq{}on\PYGZus{}enter\PYGZus{}in\PYGZus{}basket\PYGZus{}normal\PYGZdq{}}\PYG{n+nt}{/\PYGZgt{}}
            \PYG{n+nt}{\PYGZlt{}onchange} \PYG{n+na}{target=}\PYG{l+s}{\PYGZdq{}on\PYGZus{}change\PYGZus{}in\PYGZus{}basket\PYGZus{}normal\PYGZdq{}}\PYG{n+nt}{/\PYGZgt{}}
            \PYG{n+nt}{\PYGZlt{}onexit} \PYG{n+na}{target=}\PYG{l+s}{\PYGZdq{}on\PYGZus{}exit\PYGZus{}in\PYGZus{}basket\PYGZus{}normal\PYGZdq{}}\PYG{n+nt}{/\PYGZgt{}}

            \PYG{n+nt}{\PYGZlt{}transition} \PYG{n+na}{event=}\PYG{l+s}{\PYGZdq{}do\PYGZus{}goto\PYGZus{}in\PYGZus{}basket\PYGZus{}normal\PYGZdq{}} \PYG{n+na}{next=}\PYG{l+s}{\PYGZdq{}in\PYGZus{}recalculation\PYGZdq{}}\PYG{n+nt}{/\PYGZgt{}}
            \PYG{n+nt}{\PYGZlt{}transition} \PYG{n+na}{event=}\PYG{l+s}{\PYGZdq{}do\PYGZus{}add\PYGZus{}to\PYGZus{}basket\PYGZdq{}} \PYG{n+na}{next=}\PYG{l+s}{\PYGZdq{}in\PYGZus{}recalculation\PYGZdq{}}\PYG{n+nt}{/\PYGZgt{}}
            \PYG{n+nt}{\PYGZlt{}transition} \PYG{n+na}{event=}\PYG{l+s}{\PYGZdq{}do\PYGZus{}remove\PYGZus{}product\PYGZdq{}} \PYG{n+na}{next=}\PYG{l+s}{\PYGZdq{}in\PYGZus{}recalculation\PYGZdq{}}\PYG{n+nt}{/\PYGZgt{}}
        \PYG{n+nt}{\PYGZlt{}/state\PYGZgt{}}

        \PYG{n+nt}{\PYGZlt{}state} \PYG{n+na}{id=}\PYG{l+s}{\PYGZdq{}in\PYGZus{}basket\PYGZus{}freeze\PYGZdq{}}\PYG{n+nt}{\PYGZgt{}}
            \PYG{n+nt}{\PYGZlt{}transition} \PYG{n+na}{event=}\PYG{l+s}{\PYGZdq{}do\PYGZus{}goto\PYGZus{}in\PYGZus{}basket\PYGZus{}freeze\PYGZdq{}} \PYG{n+na}{next=}\PYG{l+s}{\PYGZdq{}in\PYGZus{}recalculation\PYGZdq{}}\PYG{n+nt}{/\PYGZgt{}}
            \PYG{n+nt}{\PYGZlt{}transition} \PYG{n+na}{event=}\PYG{l+s}{\PYGZdq{}do\PYGZus{}add\PYGZus{}to\PYGZus{}basket\PYGZdq{}} \PYG{n+na}{next=}\PYG{l+s}{\PYGZdq{}in\PYGZus{}recalculation\PYGZdq{}}\PYG{n+nt}{/\PYGZgt{}}
            \PYG{n+nt}{\PYGZlt{}transition} \PYG{n+na}{event=}\PYG{l+s}{\PYGZdq{}do\PYGZus{}remove\PYGZus{}product\PYGZdq{}} \PYG{n+na}{next=}\PYG{l+s}{\PYGZdq{}in\PYGZus{}recalculation\PYGZdq{}}\PYG{n+nt}{/\PYGZgt{}}
        \PYG{n+nt}{\PYGZlt{}/state\PYGZgt{}}

        \PYG{n+nt}{\PYGZlt{}state} \PYG{n+na}{id=}\PYG{l+s}{\PYGZdq{}in\PYGZus{}basket\PYGZus{}empty\PYGZdq{}} \PYG{n+na}{initial=}\PYG{l+s}{\PYGZdq{}true\PYGZdq{}}\PYG{n+nt}{\PYGZgt{}}
            \PYG{n+nt}{\PYGZlt{}transition} \PYG{n+na}{event=}\PYG{l+s}{\PYGZdq{}do\PYGZus{}add\PYGZus{}to\PYGZus{}basket\PYGZdq{}} \PYG{n+na}{next=}\PYG{l+s}{\PYGZdq{}in\PYGZus{}recalculation\PYGZdq{}}\PYG{n+nt}{/\PYGZgt{}}
        \PYG{n+nt}{\PYGZlt{}/state\PYGZgt{}}
    \PYG{n+nt}{\PYGZlt{}/state\PYGZgt{}}

\PYG{n+nt}{\PYGZlt{}/scxml\PYGZgt{}}
\end{Verbatim}


\section{Базовые принципы и возможности предложенного формата}
\label{xml_format:id1}
Формат описания является подмножеством стандарта \href{http://www.w3.org/TR/scxml/}{State Chart XML (SCXML)} с некоторыми ограничениями и дополнениями. О них подробнее будет описано ниже.


\subsection{\textless{}state\textgreater{}}
\label{xml_format:state}
Тег \textbf{\textless{}state\textgreater{}} описывает состояние конечного автомата (в данном случае – ``in\_recalculation''). Формальный список атрибутов:

\begin{tabulary}{\linewidth}{|L|L|L|L|}
\hline
\textsf{\relax 
Название
} & \textsf{\relax 
Обязательный
} & \textsf{\relax 
Значения
} & \textsf{\relax 
Описание
}\\
\hline
\code{id}
 & 
да
 & 
любое корректное для Python имя, желательно
соответствующее определённому соглашению
(как вариант - приставка \emph{in\_} перед
названием состояния)
 & 
В терминах FSM, это название состояния,
которое может принимать машина
\\

\code{initial}
 & 
нет
 & 
``true'', наличие атрибута определяющее
 & 
Состояние, в которое машина переходит
сразу же после запуска
\\

\code{final}
 & 
нет
 & 
``true'', наличие атрибута определяющее
 & 
Индикатор конечного состояния машины
\\

\code{targetns}
 & 
нет
 & 
Полный путь к модулю с callback{}`ами состояния.
Используется принятый в Python формат
 & 
Используется для сокращения записей
об источнике callback{}`ов состояния
\\
\hline\end{tabulary}


Тег \textbf{\textless{}state\textgreater{}} может включать в себя:
\begin{itemize}
\item {} 
\textbf{\textless{}transition\textgreater{}} – описание возможных переходов в другие состояния (``входной алфавит'' в терминах FSM или ``список событий'' в терминах HSME)

\item {} 
\textbf{\textless{}onentry\textgreater{}} – callback, срабатывающий \emph{при попытке войти} в состояние

\item {} 
\textbf{\textless{}onchange\textgreater{}} – callback, срабатывающий \emph{во время входа} в состояние

\item {} 
\textbf{\textless{}onexit\textgreater{}} – callback, срабатывающий во время \emph{попытки выхода} из состояния

\item {} 
\textbf{\textless{}state\textgreater{}} – вложенное подсостояние, может встречаться 0 или более раз

\end{itemize}


\subsection{\textless{}transition\textgreater{}}
\label{xml_format:transition}
Тег \textbf{\textless{}transition\textgreater{}} описывает атрибутами \emph{event} и \emph{next}, по какому событию event в какое состояние next можно перейти из данного состояния:

\begin{Verbatim}[commandchars=\\\{\}]
\PYGZlt{}transition event=\PYGZdq{}do\PYGZus{}goto\PYGZus{}in\PYGZus{}basket\PYGZus{}freeze\PYGZdq{} next=\PYGZdq{}in\PYGZus{}recalculation\PYGZdq{}/\PYGZgt{}
\end{Verbatim}

Может встречаться 0 и более раз в описании состояния.


\subsection{\textless{}onentry\textgreater{}}
\label{xml_format:onentry}
Тег \textbf{\textless{}onentry\textgreater{}} описывает атрибутом \emph{target} имя функции callback{}`а на попытку входа в состояние. Значение \emph{target} – полный путь к данной функции в python-стиле ``пакет.модуль.функция''. Если в определении \emph{\textless{}state\textgreater{}} был указан атрибут \emph{targetns}, путь к функции можно опустить. Таким образом, запись:

\begin{Verbatim}[commandchars=\\\{\}]
\PYGZlt{}state id=\PYGZdq{}in\PYGZus{}basket\PYGZus{}normal\PYGZdq{} targetns=\PYGZdq{}charts.basket\PYGZus{}callbacks.in\PYGZus{}basket\PYGZus{}normal\PYGZdq{}\PYGZgt{}
    \PYGZlt{}onentry target=\PYGZdq{}on\PYGZus{}enter\PYGZus{}in\PYGZus{}basket\PYGZus{}normal\PYGZdq{}/\PYGZgt{}
\PYGZlt{}/state\PYGZgt{}
\end{Verbatim}

и:

\begin{Verbatim}[commandchars=\\\{\}]
\PYGZlt{}state id=\PYGZdq{}in\PYGZus{}basket\PYGZus{}normal\PYGZdq{}\PYGZgt{}
    \PYGZlt{}onentry target=\PYGZdq{}charts.basket\PYGZus{}callbacks.in\PYGZus{}basket\PYGZus{}normal.on\PYGZus{}enter\PYGZus{}in\PYGZus{}basket\PYGZus{}normal\PYGZdq{}/\PYGZgt{}
\PYGZlt{}/state\PYGZgt{}
\end{Verbatim}

абсолютно равнозначны.

\begin{notice}{note}{Note:}
Переход в состояние осуществляется только в случае успешной отработки логики \emph{\textless{}onentry\textgreater{}}. Тег может быть указан только 1 раз в родительском \emph{\textless{}state\textgreater{}}.
\end{notice}


\subsection{\textless{}onchange\textgreater{}}
\label{xml_format:onchange}
Тег \textbf{\textless{}onchange\textgreater{}} описывает атрибутом \emph{target} имя функции callback{}`а состояния. Значение \emph{target} – полный путь к данной функции в python-стиле ``пакет.модуль.функция''. Если в определении \emph{\textless{}state\textgreater{}} был указан атрибут \emph{targetns}, путь к функции можно опустить. Таким образом, запись:

\begin{Verbatim}[commandchars=\\\{\}]
\PYGZlt{}state id=\PYGZdq{}in\PYGZus{}basket\PYGZus{}normal\PYGZdq{} targetns=\PYGZdq{}charts.basket\PYGZus{}callbacks.in\PYGZus{}basket\PYGZus{}normal\PYGZdq{}\PYGZgt{}
    \PYGZlt{}onchange target=\PYGZdq{}on\PYGZus{}change\PYGZus{}in\PYGZus{}basket\PYGZus{}normal\PYGZdq{}/\PYGZgt{}
\PYGZlt{}/state\PYGZgt{}
\end{Verbatim}

и:

\begin{Verbatim}[commandchars=\\\{\}]
\PYGZlt{}state id=\PYGZdq{}in\PYGZus{}basket\PYGZus{}normal\PYGZdq{}\PYGZgt{}
    \PYGZlt{}onchange target=\PYGZdq{}charts.basket\PYGZus{}callbacks.in\PYGZus{}basket\PYGZus{}normal.on\PYGZus{}change\PYGZus{}in\PYGZus{}basket\PYGZus{}normal\PYGZdq{}/\PYGZgt{}
\PYGZlt{}/state\PYGZgt{}
\end{Verbatim}

абсолютно равнозначны.

\begin{notice}{note}{Note:}
Тег может быть указан только 1 раз в родительском \emph{\textless{}state\textgreater{}}.
\end{notice}


\subsection{\textless{}onexit\textgreater{}}
\label{xml_format:onexit}
Тег \textbf{\textless{}onexit\textgreater{}} описывает атрибутом \emph{target} имя функции callback{}`а на попытку выхода из состояния. Значение \emph{target} – полный путь к данной функции в python-стиле ``пакет.модуль.функция''. Если в определении \emph{\textless{}state\textgreater{}} был указан атрибут \emph{targetns}, путь к функции можно опустить. Таким образом, запись:

\begin{Verbatim}[commandchars=\\\{\}]
\PYGZlt{}state id=\PYGZdq{}in\PYGZus{}basket\PYGZus{}normal\PYGZdq{} targetns=\PYGZdq{}charts.basket\PYGZus{}callbacks.in\PYGZus{}basket\PYGZus{}normal\PYGZdq{}\PYGZgt{}
    \PYGZlt{}onexit target=\PYGZdq{}on\PYGZus{}exit\PYGZus{}in\PYGZus{}basket\PYGZus{}normal\PYGZdq{}/\PYGZgt{}
\PYGZlt{}/state\PYGZgt{}
\end{Verbatim}

и:

\begin{Verbatim}[commandchars=\\\{\}]
\PYGZlt{}state id=\PYGZdq{}in\PYGZus{}basket\PYGZus{}normal\PYGZdq{}\PYGZgt{}
    \PYGZlt{}onexit target=\PYGZdq{}charts.basket\PYGZus{}callbacks.in\PYGZus{}basket\PYGZus{}normal.on\PYGZus{}exit\PYGZus{}in\PYGZus{}basket\PYGZus{}normal\PYGZdq{}/\PYGZgt{}
\PYGZlt{}/state\PYGZgt{}
\end{Verbatim}

абсолютно равнозначны.

\begin{notice}{note}{Note:}
В случае неудачи отработки логики \emph{\textless{}onexit\textgreater{}}, выхода из состояния не происходит. Тег может быть указан только 1 раз в родительском \emph{\textless{}state\textgreater{}}.
\end{notice}


\subsection{\textless{}state\textgreater{}}
\label{xml_format:id2}
Вложенные состояния могут использоваться для визуальной группировки по логическому принципу. Состояние, которое включает в себя вложенные состояния является мета-состоянием. Помимо структурной (визуальной) группировки на уровне описания, для meta-состояния создаётся набор служебных событий для переходов во вложенные состояния.

Для мета-состояния ``in\_recalculation'', на этапе парсинга во внутреннее представление, создаются служебные события вида \emph{do\_goto\_} + \emph{имя\_вложенного\_состояния}. Записи вида:

\begin{Verbatim}[commandchars=\\\{\}]
\PYGZlt{}state id=\PYGZdq{}in\PYGZus{}recalculation\PYGZdq{} targetns=\PYGZdq{}charts.basket\PYGZus{}callbacks.in\PYGZus{}recalculation\PYGZdq{}\PYGZgt{}
    \PYGZlt{}onentry target=\PYGZdq{}on\PYGZus{}enter\PYGZus{}in\PYGZus{}recalculation\PYGZdq{}/\PYGZgt{}
    \PYGZlt{}onchange target=\PYGZdq{}on\PYGZus{}change\PYGZus{}in\PYGZus{}recalculation\PYGZdq{}/\PYGZgt{}
    \PYGZlt{}onexit target=\PYGZdq{}on\PYGZus{}exit\PYGZus{}in\PYGZus{}recalculation\PYGZdq{}/\PYGZgt{}

    \PYGZlt{}state id=\PYGZdq{}in\PYGZus{}basket\PYGZus{}normal\PYGZdq{} targetns=\PYGZdq{}charts.basket\PYGZus{}callbacks.in\PYGZus{}basket\PYGZus{}normal\PYGZdq{}\PYGZgt{}
        \PYGZlt{}onentry target=\PYGZdq{}on\PYGZus{}enter\PYGZus{}in\PYGZus{}basket\PYGZus{}normal\PYGZdq{}/\PYGZgt{}
        \PYGZlt{}onchange target=\PYGZdq{}on\PYGZus{}change\PYGZus{}in\PYGZus{}basket\PYGZus{}normal\PYGZdq{}/\PYGZgt{}
        \PYGZlt{}onexit target=\PYGZdq{}on\PYGZus{}exit\PYGZus{}in\PYGZus{}basket\PYGZus{}normal\PYGZdq{}/\PYGZgt{}

        \PYGZlt{}transition event=\PYGZdq{}do\PYGZus{}goto\PYGZus{}in\PYGZus{}basket\PYGZus{}normal\PYGZdq{} next=\PYGZdq{}in\PYGZus{}recalculation\PYGZdq{}/\PYGZgt{}
        \PYGZlt{}transition event=\PYGZdq{}do\PYGZus{}add\PYGZus{}to\PYGZus{}basket\PYGZdq{} next=\PYGZdq{}in\PYGZus{}recalculation\PYGZdq{}/\PYGZgt{}
        \PYGZlt{}transition event=\PYGZdq{}do\PYGZus{}remove\PYGZus{}product\PYGZdq{} next=\PYGZdq{}in\PYGZus{}recalculation\PYGZdq{}/\PYGZgt{}
    \PYGZlt{}/state\PYGZgt{}

    \PYGZlt{}state id=\PYGZdq{}in\PYGZus{}basket\PYGZus{}empty\PYGZdq{} initial=\PYGZdq{}true\PYGZdq{}\PYGZgt{}
        \PYGZlt{}transition event=\PYGZdq{}do\PYGZus{}add\PYGZus{}to\PYGZus{}basket\PYGZdq{} next=\PYGZdq{}in\PYGZus{}recalculation\PYGZdq{}/\PYGZgt{}
    \PYGZlt{}/state\PYGZgt{}

\PYGZlt{}/state\PYGZgt{}
\end{Verbatim}

и:

\begin{Verbatim}[commandchars=\\\{\}]
\PYGZlt{}state id=\PYGZdq{}in\PYGZus{}recalculation\PYGZdq{} targetns=\PYGZdq{}charts.basket\PYGZus{}callbacks.in\PYGZus{}recalculation\PYGZdq{}\PYGZgt{}
    \PYGZlt{}onentry target=\PYGZdq{}on\PYGZus{}enter\PYGZus{}in\PYGZus{}recalculation\PYGZdq{}/\PYGZgt{}
    \PYGZlt{}onchange target=\PYGZdq{}on\PYGZus{}change\PYGZus{}in\PYGZus{}recalculation\PYGZdq{}/\PYGZgt{}
    \PYGZlt{}onexit target=\PYGZdq{}on\PYGZus{}exit\PYGZus{}in\PYGZus{}recalculation\PYGZdq{}/\PYGZgt{}

    \PYGZlt{}transition event=\PYGZdq{}do\PYGZus{}goto\PYGZus{}in\PYGZus{}basket\PYGZus{}normal\PYGZdq{} next=\PYGZdq{}in\PYGZus{}recalculation\PYGZdq{}/\PYGZgt{}
    \PYGZlt{}transition event=\PYGZdq{}do\PYGZus{}goto\PYGZus{}in\PYGZus{}basket\PYGZus{}empty\PYGZdq{} next=\PYGZdq{}in\PYGZus{}recalculation\PYGZdq{}/\PYGZgt{}

\PYGZlt{}/state\PYGZgt{}

\PYGZlt{}state id=\PYGZdq{}in\PYGZus{}basket\PYGZus{}normal\PYGZdq{} targetns=\PYGZdq{}charts.basket\PYGZus{}callbacks.in\PYGZus{}basket\PYGZus{}normal\PYGZdq{}\PYGZgt{}
    \PYGZlt{}onentry target=\PYGZdq{}on\PYGZus{}enter\PYGZus{}in\PYGZus{}basket\PYGZus{}normal\PYGZdq{}/\PYGZgt{}
    \PYGZlt{}onchange target=\PYGZdq{}on\PYGZus{}change\PYGZus{}in\PYGZus{}basket\PYGZus{}normal\PYGZdq{}/\PYGZgt{}
    \PYGZlt{}onexit target=\PYGZdq{}on\PYGZus{}exit\PYGZus{}in\PYGZus{}basket\PYGZus{}normal\PYGZdq{}/\PYGZgt{}

    \PYGZlt{}transition event=\PYGZdq{}do\PYGZus{}goto\PYGZus{}in\PYGZus{}basket\PYGZus{}normal\PYGZdq{} next=\PYGZdq{}in\PYGZus{}recalculation\PYGZdq{}/\PYGZgt{}
    \PYGZlt{}transition event=\PYGZdq{}do\PYGZus{}add\PYGZus{}to\PYGZus{}basket\PYGZdq{} next=\PYGZdq{}in\PYGZus{}recalculation\PYGZdq{}/\PYGZgt{}
    \PYGZlt{}transition event=\PYGZdq{}do\PYGZus{}remove\PYGZus{}product\PYGZdq{} next=\PYGZdq{}in\PYGZus{}recalculation\PYGZdq{}/\PYGZgt{}
\PYGZlt{}/state\PYGZgt{}

\PYGZlt{}state id=\PYGZdq{}in\PYGZus{}basket\PYGZus{}empty\PYGZdq{} initial=\PYGZdq{}true\PYGZdq{}\PYGZgt{}
    \PYGZlt{}transition event=\PYGZdq{}do\PYGZus{}add\PYGZus{}to\PYGZus{}basket\PYGZdq{} next=\PYGZdq{}in\PYGZus{}recalculation\PYGZdq{}/\PYGZgt{}
\PYGZlt{}/state\PYGZgt{}
\end{Verbatim}

абсолютно равнозначны.

\begin{notice}{note}{Note:}
Callback{}`и и пространство имён callback{}`ов мета-состояния не распространяются на вложенные состояния. Описания тегов \textless{}onchange\textgreater{}, \textless{}onentry\textgreater{}, \textless{}onexit\textgreater{} и state-атрибута targetns не наследуются.
\end{notice}
\begin{itemize}
\item {} 
\emph{genindex}

\item {} 
\emph{search}

\end{itemize}



\renewcommand{\indexname}{Index}
\printindex
\end{document}
